
\documentclass[12pt,a4paper,draft]{article}
%setting margins
\usepackage[margin=0.75in]{geometry}


%%%%%%%%%%%%%%%%%%%%%%%%%%%%%%%%%%%%%%%%%%%%%%%%%%%%%%%%
%packages
\usepackage{amsmath}
\usepackage{amsthm}
\usepackage{blindtext}
\usepackage{fancyhdr}
\usepackage{enumerate}
\usepackage[shortlabels]{enumitem}
\usepackage{setspace}
\usepackage{tcolorbox}
\usepackage{pgfplots}
\usepackage{multicol}
\usepackage{subfiles} % Best loaded last in the preamble
\pgfplotsset{compat=1.18}
\usepackage{subcaption}
\usepackage{graphicx}
\usepackage{tabularray}
\usepackage{circuitikz}

\usepackage[thinc]{esdiff}
%%%%%%%%%%%%%%%%%%%%%%%%%%%%%%%%%%%%%%%%%%%%%%%%%%%%%%%%
\pagestyle{fancy}
\fancyhead[l]{Rüzgar Erik}
\fancyhead[c]{Notes For the 14th Unit}
\fancyhead[r]{MATH104}
\fancyfoot[c]{\thepage}
\renewcommand{\headrulewidth}{0.2pt}
\setlength{\headheight}{15pt}
%%%%%%%%%%%%%%%%%%%%%%%%%%%%%%%%%%%%%%%%%%%%%%%%%%%%%%%%
%makes all equations centered use originalequation for normal mode
\let\originalequation\equation
\let\endoriginalequation\endequation

\renewenvironment{equation}
  {\begin{center}\originalequation}
  {\endoriginalequation\end{center}}


%%%%%%%%%%%%%%%%%%%%%%%%%%%%%%%%%%%%%%%%%%%%%%%%%%%%%%%%
%Define a new tcolorbox environment named "definition"
  \newtcolorbox{definitionbox}{
    colback=blue!5!white,
    colframe=blue!75!black,
    title=DEFINITION(S)
  }
  
  \newenvironment{definition}{\begin{definitionbox}}{\end{definitionbox}\vspace{1\baselineskip}}

%%%%%%%%%%%%%%%%%%%%%%%%%%%%%%%%%%%%%%%%%%%%%%%%%%%%%%%%
\newcommand{\example}{\vspace{1\baselineskip}\par\noindent\textcolor{red}{\textbf{EXAMPLE}: }}
\newcommand{\solution}{\par\noindent\textcolor{red}{\textbf{SOLUTION}: }\vspace{1\baselineskip}}

%%%%%%%%%%%%%%%%%%%%%%%%%%%%%%%%%%%%%%%%%%%%%%%%%%%%%%%%

%Define a new tcolorbox environment named "theorem"
\newtcolorbox{theorembox}[1]{
    colback=red!5!white,
    colframe=red!75!black,
    title=THEOREM
  }
  
  \newenvironment{theorem}{\begin{theorembox}}{\end{theorembox}\vspace{1\baselineskip}}

%%%%%%%%%%%%%%%%%%%%%%%%%%%%%%%%%%%%%%%%%%%%%%%%%%%%%%%%

\newtcolorbox{corollarybox}{
    colback=blue!5!white,
    colframe=blue!75!black,
    title=Corollary
}
  
\newenvironment{corollary}{\begin{corollarybox}}{\end{corollarybox}\vspace{1\baselineskip}}


\newtcolorbox{rulebox}[1]{
    colback=red!5!white,
    colframe=red!75!black,
    title=#1
}

\newenvironment{ruleBox}[1]{\begin{rulebox}{#1}}{\end{rulebox}\vspace{1\baselineskip}}


%%%%%%%%%%%%%%%%%%%%%%%%%%%%%%%%%%%%%%%%%%%%%%%%%%%%%%%%

%Define a new tcolorbox environment named "theorem"

\newtcolorbox{note}{
    colback=white,
    colframe=black,
}

\newenvironment{mynote}{\vspace{1\baselineskip}\begin{note}}{\end{note}\vspace{1\baselineskip}}

%%%%%%%%%%%%%%%%%%%%%%%%%%%%%%%%%%%%%%%%%%%%%%%%%%%%%%%%



\newcommand{\lopital}{L'Hôpital's Rule }
\newcommand{\anc}{\(\{a_n\}\)}
\newcommand{\fxy}{\(f(x,y)\)}
\newcommand{\an}{\(a_n\)}

\newcommand{\infsum}{\sum_{n=1}^{\infty}}
\fancypagestyle{firstpage}{
    \fancyhf{} % Clear header and footer
    \renewcommand{\headrulewidth}{0pt} % Remove header rule
}

% Redefine \maketitle to include a fancy box
\makeatletter
\renewcommand{\maketitle}{%
  \thispagestyle{firstpage} % Apply the firstpage style for the first page
  \begin{tcolorbox}[colback=white,colframe=black,width=\textwidth,arc=0mm,auto outer arc]
    \begin{center}
      \Large \@title \\[1ex] \large \@date \\ \textit{Rüzgar ERİK}
    \end{center}
  \end{tcolorbox}
}
\makeatother

\usepackage{parskip}
\setlength{\parindent}{0pt} % Disable paragraph indentation


\usepackage{hyperref}
\hypersetup{
    colorlinks,
    citecolor=black,
    filecolor=black,
    linkcolor=black,
    urlcolor=black
}


%%%%%%%%%%%%%%%%%%%%%%%%%%%%%%%%%%%%%%%%%%%%%%%%%%%%%%%%

\begin{document}



\title{Multiple Integrals}
\date{MATH 104}
\maketitle
\tableofcontents % This command generates the index page

\newpage % Start a new page for the content

\section{Double and Iterated Integrals over Rectangles}

\subsection{Double Integrals}

We begin our investigations of double integrals by considering the simplest type of region, a rectangle. We consider a function \(f(x,y)\) defined on a rectangular region R,

\[R: \quad a \leq x \leq b, \ c \leq y \leq d\].

We subdivide \textit{R} into small rectangles using a network of lines parallel to the x- and y-axes.
The lines divide R into n rectangular pieces, where the number of such pieces n gets large as the width and height of each piece gets small. These rectangles form a \textbf{partition} of \textit{R}. 
A small rectangular piece of width \(\Delta x\) and height \(\Delta y\) has area \(\Delta A = \Delta x \Delta y\). If we number the small pieces partitioning \textit{R} in some order, then their areas are given by numbers \(\Delta A_1, \Delta A_2 , \dots , \Delta A_n\) where \(\Delta A_k\) is the area of the kth small rectangle.


To form a Riemann sum over \textit{R}, we choose a point \((x_k,y_k)\) in the kth small rectangle, multiply the value of \(f\) at that point by the area \(\Delta A_k\), and add together the products:

\[S_n = \sum_{k=1}^{n}f(x_k,y_k)\Delta A_k.\]

Depending on how we pick \((x_k,y_k)\) in the kth small rectangle, we may get different values for \(S_n\).


We are interested in what happens to these Riemann sums as the widths and heights of all the small rectangles in the partition of R approach zero. The \textbf{norm} of  a partition \textit{P} is written \(\Vert R \Vert\), is the largest width or height of any rectangle in the partition.
If \(\Vert R \Vert = 0.1\) then all the rectangles in the partition of R have width at most \(0.1\) and height at most \(0.1\).  Sometimes the Riemann sums converge as the norm of \textit{\textbf{R}} goes to zero, written \(\Vert P \Vert  \to 0\). 

The resulting limit is written as
\[\lim_{\Vert P \Vert \to 0} \sum_{k=1}^{n}f(x_k,y_k)\Delta A_k\]

as \(\Vert P \Vert  \to 0\) and the rectangles get narrow and short, their number n increases, so we can also write the limit as
\[\lim_{n \to \infty} \sum_{k=1}^{n}f(x_k,y_k)\Delta A_k\]

with understanding that  \(\Vert P \Vert  \to 0\), and hence \(\Delta A_k \to 0\) as \(n \to \infty\).

When a limit of the sums \(S_n\) exists, giving the same limiting value no matter what choices are made, then the function \(f\) is said to be \textbf{integrable} and the limit is called the \textbf{double integral} of \(f\) over \textit{R}, written as
\[\int \int_{R}^{} f(x,y) \mathrm{d}A \quad \mathrm{or} \quad \int \int_{R}^{} f(x,y) \mathrm{d}x \mathrm{d}y \]



\subsection{Double Integrals as Volumes}

When \(f(x,y)\) is a positive function over a rectangular area \textit{R} in the xy-plane, we may interpret the double integral of \(f\) over \textit{R} as the volume of the 3-dimensional solid region over the xy-plane bounded below by \textit{R}  and above by the surface \(z = f(x,y)\). Each term \(f(X_K,y_k)\Delta A_k \) in the sum  \(S_n = \sum f(x_k,y_k)\Delta A_k\) is the volume of a vertical rectangular box that approximates the volume of the portion of the solid that stands directly above the base \(\Delta A_k\). The sum \(S_n\) thus approximates what we want to call the total volume of the solid that stands directly above the base \(\Delta A_k\).  


\subsection{Fubini's Theorem for Calculating Double Integrals}

Suppose that we wish to calculate the volume under the plane \(z = 4 - x -y\) over the rectangular region \(R: 0 \leq x \leq 2, 0 \leq y \leq 1\) in the xy-plane. If we apply the method of slicing, with slices perpendicular to the x-axis, then the volume is

\[\int_{x=0}^{x=2}A(x)\mathrm{d}x\],

where \(A(x)\) is the cross-sectional area at x. For each value of x, we may calculate \(A(x)\) as the integral

\[A(X) = \int_{y=0}^{y=1}(4 - x - y)\mathrm{d}y\],

which is the area under the curve in the plane of the cross-section at x. In calculating \(A(x)\), x is held fixed and the integration takes place with respect to y. Combining the equations the volume of the entire solid is
\[\text{Volume}= \ \int_{0}^{2} \int_{0}^{1}(4 - x - y) \mathrm{d}y\mathrm{d}x\].


\begin{ruleBox}{Fubini's Theorem First Form}

    If \(f(x,y)\) is continuous throughout the rectangular region \(R: a \leq x \leq b, c \leq y \leq d\) then

    \[\iint_R f(x, y) d A=\int_c^d \int_a^b f(x, y) d x d y=\int_a^b \int_c^d f(x, y) d y d x\]

\end{ruleBox}

\section{Double Integrals over General Regions}

\subsection{Double Integrals over Bounded, Nonrectangular Regions}
\begin{ruleBox}{Fubini's Theorem (Stronger Form)}

    Let \(f(x,y)\) be continuous on a region R. 

    \begin{enumerate}
        \item If R is defined by \(a \leq x \leq b, g_1(x) \leq y g_2(x)\), with \(g_1\) and \(g_2\) continuous on \([a,b]\) then, \[\iint_R f(x, y) d A=\int_a^b \int_{g_1(x)}^{g_2(x)} f(x, y) d y d x .\]
        \item If R is defined by \(c \leq y \leq d, h_1(y) \leq x \leq h_2(y)\), with \(h_1\) and \(h_2\) continuous on \([c,d]\), then \[\iint_R f(x, y) d A=\int_c^d \int_{h_1(y)}^{h_2(y)} f(x, y) d x d y\]
    \end{enumerate}
    
\end{ruleBox}

\subsection{Finding the Limits of Integration}

\textit{\textbf{Using Vertical Cross-Sections}}
\begin{enumerate}
    \item Sketch. Sketch the region of integration and label the bounding curves.
    \item Find the y-limits of integration. Imagine a vertical line cuttting through R in the direction of increasing y. Mark the y-values where L enters and leaves. These are y-limits of integration and are usually functions of x.
    \item \textit{Find the x-limits of the integration.} Choose x-limits that include all the vertical lines through R. The integral shown in the figure is
    

\end{enumerate}




\subsection{Properties of Double Integrals}



\begin{mynote}

If \fxy and \(g(x,y)\) are continuous on the bounded region \textit{R}, then the following properties hold.

\begin{enumerate}
    \item Constant Multiple: \[\iint_R c f(x,y) \mathrm{d}A = c \cdot \iint_R  f(x,y) \mathrm{d}A\]
    \item Sum and difference: \[\iint_R (f(x,y) \pm g(x,y)) \mathrm{d}A = \iint_R f(x,y) \mathrm{d}A  \pm \iint_R g(x,y) \mathrm{d}A\]
    \item Domination: 
    \begin{enumerate}[label=\alph*)]
        \item \[\iint_R f(x,y) \mathrm{d}A \geq 0 \quad \mathrm{if} \quad f(x,y) \geq 0 \ on \  R\]
        \item \[\iint_R f(x,y) \mathrm{d}A \geq \iint_R g(x,y) \mathrm{d}A \quad \mathrm{if} \quad f(x,y) \geq g(x,y) \ on \ R\]
    \end{enumerate}
    \item Additivity: If \textit{R} is the union of two non overlapping Regions \(R_1\) and \(R_2\), then \[\iint_R f(x,y) \mathrm{d}A =\iint_{R_1} f(x,y) \mathrm{d}A + \iint_{R_2} f(x,y) \mathrm{d}A\]
\end{enumerate}


\end{mynote}
  

\section{Area by Double Integration}


\subsection{Areas of Bounded Regions in the Plane}

If we take \(f(x,y) = 1\) in the definition of the double integral over a region \textit{R} in the preceding section, the Riemann sums reduce to 
\[S_n = \sum_{k=1}^{n} f(x_k,y_k) \Delta A_k = \sum_{k=1}^{n} \Delta A_k.\]

\begin{definition}
    The \textbf{area} of a closed, bounded plane region \textit{R} is 

    \[A = \iint_R \mathrm{d}A\]
\end{definition}


\subsection{Average Value}


\begin{mynote}
    \textbf{Average Value } of \(f\) over \textit{R} \[\frac{1}{\text{area of R}  } \iint_R f \mathrm{d}A\] 


\end{mynote}

\section{Double Integrals In Polar Form}


\subsection{Integrals In Polar Coordinates}


When we defined the double integral of a function over a region \textit{R} in the xy-plane, we began by cutting \textit{R} into rectangles whose sides were parallel to the coordinate axes. These were the natural shapes to use because their sides have either constant x-values or constant y-values. In polar coordinates, the natural shape is "polar rectangle" whose sides have constant r- and \(\theta\)-values.
To avoid ambiguities when describing the region of integration with polar coordinates, we use polar coordinate points \((r, \theta)\) where \(r \geq 0\).

Suppose a function \(f(r, \theta)\) is defined over a region \textit{R} that is bounded by the rays \(\theta = \alpha\) and \(\theta = \beta\) and by the continuous curves \(r = g_1(\theta)\) and \(r = g_2(\theta)\).
Suppose also that \(0 \leq g_1(\theta) \leq g_2(\theta) \leq \alpha\) for every value of \(\theta\) between \(\alpha\) and \(\beta\). Then \textit{R} lies in a fan shaped region Q defined by the inequalities \(0 \leq r \leq a\) and \(\alpha \leq \theta \leq \beta\) where \(0 \leq \beta - \alpha \leq 2 \pi\)


We cover Q by a grid of circular arcs and rays. The arcs are cut from circles centered at the origin, with radii \(\Delta r, \Delta 2r, \dots , m \Delta r\) where \(\Delta r = \frac{a}{m}\).The rays are given by
\[\theta = \alpha, \quad \theta = \alpha + \Delta \theta, \quad \theta = \alpha + 2 \Delta \theta , \quad \dots, \theta = \alpha + m' \Delta \theta = \beta\]

where \(\Delta \theta = (\beta - \alpha) / m'\). The arcs and rays partition Q into smaller patches called called "polar rectangles" 

We number the polar rectangles that lie inside \textit{R}, calling their areas \(\Delta A_1, \Delta A_2 , \dots , \Delta A_n\) where \(\Delta A_n\). We let \((r_k, \theta_k)\) be any point in the polar rectangle whose area is \(\Delta A_k\). We form the sum

\[S_n = \sum_{k=1}^{n} f(r_k,\theta_k) \Delta A_k\].

If f is continuous throughout \textit{R}, this sum will approach a limit as we define the grid to make \(\Delta r\) and \(\Delta \theta\) go to zero. The limit is called the double integral of \(f\) over \textit{R}. In symbols,

\[\lim_{n \to \infty} S_n = \iint_R f(r, \theta) \mathrm{d}A\]


To evaluate this limit, we first have to write the sum \(S_n\) in a way that expresses \(\Delta A_k\) in terms of \(\Delta R\) and \(\Delta \theta\). For convenience we choose \(r_k\) to be the average of the radii of the inner and outer arcs bounding the kth polar rectangle \(\Delta A_k\). The radius of the inner arc bounding \(\Delta A_k\) is then \(r_k - (\Delta r / 2)\).
The radius of the outer arc is \(r_k + (\Delta r / 2)\)

The area of a wedge-shaped sector of a circle having radius r and angle \(\theta\) is 

\[A = \frac{1}{2} \theta r^2\]

as can be seen by multiplying \(\pi r^2\), the area of the circle, by \(\theta / 2 \pi\), the fraction of the circle's area contained in the wedge. So the areas of the circular sectors subtended by these arcs at the origin are

\begin{align*}
    \text{Inner Radius: } \quad \frac{1}{2} \left( r_k - \frac{\Delta r}{2}\right)^2 \Delta \theta \\
    \\
    \text{Inner Radius: } \quad \frac{1}{2} \left( r_k + \frac{\Delta r}{2}\right)^2 \Delta \theta 
\end{align*}

Therefore 


\begin{align*}
    \Delta A_k &= \text{area of the large sector} - \text{area of the small sector}\\
    &=\frac{\Delta \theta}{2}\left[\left(r_k+\frac{\Delta r}{2}\right)^2-\left(r_k-\frac{\Delta r}{2}\right)^2\right]=\frac{\Delta \theta}{2}\left(2 r_k \Delta r\right)=r_k \Delta r \Delta \theta.
\end{align*}


A version of Fubini's Theorem says that the limit approached by these sums can be evaluated by repeated single integrations with respect tot r and \(\theta\) as

\[\iint_R f(r, \theta) dA = \int_{\theta = \alpha}^{\theta = \beta} \int_{r = g_1(\theta)}^{g_2(\theta)}f(r,\theta) r dr d\theta\]




\subsection{Finding Limits of Integration}

\begin{enumerate}
    \item Sketch. Sketch the region and label the bounding curves.
    \item Find the r limits of the integration. 
    \item Find \(\theta\)-limits of the integration. Find the smallest and largest \(\theta\)-values that bound R. 
\end{enumerate}





\begin{mynote}
    \textcolor{blue}{\textbf{Area in Polar Coordinates}} \\
    The area of a a closed and bounded region R in the polar coordinate plane is
    \[A = \iint_R r dr d \theta\]
\end{mynote}


\newpage

\subsection{Changing Cartesian Integrals into Polar Integrals}

The procedure for changing a Cartesian integral into a polar integral has two steps. First substitute \(x = r \cos \theta\) and \(y = r \sin \theta\) and replace \(\mathrm{d}x \mathrm{d}y\) by \(r \mathrm{d}r \mathrm{d}\theta\) in the cartesian integral. Then supply polar limits of integration for the boundry of R. The cartesian integral becomes

\[\iint_R f(x,y) dx dy = \iint_G f(r \cos \theta, r \sin \theta) r dr d\theta\]




\section{Triple Integrals in Rectangular Coordinates}

\section{Volume of a Region in Space}

If F is the constant function whose value is 1, then the sums reduce to

\[S_n = \sum_{k=1}^{n} F(x_k, y_k ,z_k) \Delta V_k = \sum_{k=1}^{n} 1 \cdot \Delta V_k = \sum_{k=1}^{n}  \Delta V_k \] 




As \(\Delta x_k\), \(\Delta y_k\), and \(\Delta z_k\) approach zero, the cells \(\Delta V_k\) become small and more numerous and fill up more and more of D. We therefore define the volume of D to be the triple integral.

\[\lim_{n \to \infty} \Delta V_k = \iiiint_D dV\]



\begin{definition}
    The \textbf{volume} of a solid region D in space is
    \[V = \iiint_D \mathrm{d}V\]
\end{definition}


\subsection{Finding limits of Integration in the order dz dy dx}

We evaluate a triple integral by applying a three-dimensional version of Fubini's Theorem to evaluate it by three repeated single integrations. As with double integrals, there is a geometric procedure for finding the limits of the integration for these integrals.

To evaluate 
\[\iiint_D F(x,y,z) dV\]

over a region D, integrate first with respect to z, then with respect to y, and finally with respect to x. The limits of integration are found by projecting D onto the xy-plane and then projecting the resulting region onto the x-axis.

\begin{enumerate}
    \item Sketch. Sketch the region D along with its shadow R in the xy-plane. Label the upper and lower bounding surfaces of D and the bounding curves of R.
    \item Find the z-limits of the integration. Draw a line M passing through a typical point \((x,y)\) in R parallel to the z-axis. As z increases, M enters D at \(z = f_1(x,y)\) and leaves at \(z = f_2(x,y)\). These are the z-limits of integration.
    \item Find the y-limits of the integration. For each value of x, the line segment in R that lies above x is bounded below by the curve \(y = g_1(x)\) and above by \(y = g_2(x)\). These are the y-limits of integration.
    \item Find the x-limits of the integration. Choose x-limits that include all the vertical lines through R. The integral is \[ \int_{x=a}^{x=b}\int_{y=g_1(x)}^{y=g_2(x)}\int_{z = f_1(x,y)}^{z = f_2(x,y)} F(x,y,z) dz dy dx\]
\end{enumerate}

\subsection{Average Value of a Function in Space}

The average value of a function \(F(x,y,z)\) over a solid region D is

\begin{mynote}
    \[\frac{1}{\text{Volume of D}} \iiint_D F(x,y,z) dV\]
\end{mynote}



\end{document}